%=======================
% PACKAGES
%=======================
\usepackage{graphicx}
\usepackage{blindtext}
\usepackage[reqno]{amsmath}
\usepackage{amssymb, amscd, amsthm, amstext}
\usepackage{array}
\usepackage{enumitem}
\usepackage{geometry}
\usepackage{caption}
\usepackage{multirow}
\usepackage{lipsum}
\usepackage[T1]{fontenc}
\usepackage{imakeidx}
\usepackage{systeme}
\usepackage[dvipsnames]{xcolor}
\usepackage{thmtools}
\usepackage[most]{tcolorbox}
\usepackage{titlesec}
\usepackage{float}
\usepackage{bigdelim} 
\usepackage{tikz}
\usepackage{pdfpages}
\usepackage{hyperref}
\usepackage{mathtools}
\usepackage{cancel}
\usepackage{fancyvrb}
\usepackage{listings}
%=======================
% PACKAGE SETTINGS
%=======================
\hypersetup{
    colorlinks=true,
    linkcolor=blue,
    urlcolor=blue,
    citecolor=blue
}
\usepackage{bookmark}

%=======================
% NEW COMMANDS
%=======================

% Symbols
\newcommand\R{\ensuremath{\mathbb{R}}}
\newcommand\Z{\ensuremath{\mathbb{Z}}}
\renewcommand\O{\ensuremath{\emptyset}}
\newcommand\B{\ensuremath{\mathcal{B}\ }}

% Mathematical notation
\newcommand{\inv}{^{-1}}

% Redefining symbols
\let\svlim\lim \def\lim{\svlim\limits}
\let\implies\Rightarrow
\let\impliedby\Leftarrow
\let\iff\Leftrightarrow
\let\epsilon\varepsilon

\newcommand\Tstrut{\rule{0pt}{2.6ex}}         % = `top' strut
\newcommand\Bstrut{\rule[-0.9ex]{0pt}{0pt}}   % = `bottom' strut
%=======================
% STYLING AND FORMATTING
%=======================
    % Define theorem environments

    % \newtheoremstyle{name}
    %   {space above}     % 1
    %   {space below}     % 2  
    %   {body font}       % 3
    %   {indent amount}   % 4
    %   {theorem head font} % 5
    %   {punctuation after theorem head} % 6
    %   {space after theorem head} % 7
    %   {theorem head spec} % 8
    %   {space after theorem head} % 9

% Theorem

% Definition
% \newtheoremstyle{DefStyle}
%     {10pt}
%     {10pt}%
%     {}
%     {}%
%     {\bfseries}
%     {:}%
%     { }
%     {}%
%     {}
    
% Note
\newtheoremstyle{DefStyle}
    {10pt}
    {10pt}%
    {}
    {}%
    {\bfseries}
    {:}%
    { }
    {}%
    {}
    

\theoremstyle{DefStyle}
% \theoremstyle{plain}
% \theoremstyle{definition}


\newtheorem{thm}{Theorem}
\newtheorem{exmp}{Example}
\newtheorem{defn}{Definition}
\newtheorem{note}{Note}

% \declaretheoremstyle[
%     headfont=\bfseries\sffamily, bodyfont=\normalfont,
%     % mdframed={
%     %     linewidth=2pt,
%     %     rightline=false, topline=false, bottomline=false,
%     %     linecolor=ForestGreen, backgroundcolor=ForestGreen!5,
%     % }
% ]{sdef}


\declaretheoremstyle[
    headfont=\bfseries\sffamily\color{ForestGreen!70!black}, bodyfont=\normalfont,
    mdframed={
        linewidth=2pt,
        rightline=false, topline=false, bottomline=false,
        linecolor=ForestGreen, backgroundcolor=ForestGreen!5,
    }
]{thmgreenbox}

\declaretheorem{Theorem}
% \declaretheorem[style=sdef]{sdef}
\declaretheorem[style=thmgreenbox]{greenboxtheorem}


\newtcolorbox{noot}[3][]
{
  arc=0mm,
  colback  = white,
  colframe = #2!50,
  coltitle = #2!20!black,   
  title    = {Side-note: #3},
  fonttitle=\sffamily,
  breakable,
  #1,
}

\newenvironment{sidenote}[1]{\begin{noot}{gray}{#1}}{\end{noot}}

\titleformat{\chapter}[frame]
  {\normalfont}
  {\filright
   \footnotesize
   \enspace ch. \arabic{chapter}.\enspace}
  {8pt}
  {\Large\bfseries\filcenter}
\usepackage[dotinlabels]{titletoc}
\titlecontents{chapter}[1.5em]{}{\contentslabel{2.3em}}{\hspace*{-2.3em}}{\hfill\contentspage}
\titlespacing*{\chapter} {0pt}{0pt}{40pt}     % this alters "before" spacing (the second length argument) to 0

\newcommand{\nchapter}[2]{%
    \setcounter{chapter}{#1}%
    \addtocounter{chapter}{-1}%
    \chapter{#2}
}

\newcommand{\nsection}[3]{%
    \setcounter{chapter}{#1}%
    \setcounter{section}{#2}%
    \addtocounter{section}{-1}%
    \section{#3}
}%

%%%%%%%%%%%%%%%%%%%%%%%%%%%%%%%%%%%%%%%%%%%%%%%%
\usetikzlibrary{decorations.pathreplacing,calc}
\newcommand{\ntikzmark}[2]{#2\hspace{0.5em}\tikz[overlay,remember picture,baseline=(#1.base)]{\node[inner sep=0pt] (#1) {};}}

\newcommand{\makebrace}[3]{%
    \begin{tikzpicture}[overlay, remember picture]
        \draw [decoration={brace,amplitude=0.5em},decorate]
        let \p1=(#1), \p2=(#2) in
        ({max(\x1,\x2)}, {\y1+0.8em}) -- node[right=0.6em] {#3} ({max(\x1,\x2)}, {\y2});
    \end{tikzpicture}
}

%%%%%%%%%%%%%%%%%%%%%%%%%%%%%%%%%%%%%%%%%%%%%%%%
\newenvironment{idef}[2]{%
    \par%
    \vspace{0.5em}%
    \noindent\textbf{#1:} \index{#2}
}{%
    \par%
    \vspace{0.5em}%
}
