\documentclass{report}
%=======================
% PACKAGES
%=======================
\usepackage{graphicx}
\usepackage{blindtext}
\usepackage[reqno]{amsmath}
\usepackage{amssymb, amscd, amsthm, amstext}
\usepackage{array}
\usepackage{enumitem}
\usepackage{geometry}
\usepackage{caption}
\usepackage{multirow}
\usepackage{lipsum}
\usepackage[T1]{fontenc}
\usepackage{imakeidx}
\usepackage{systeme}
\usepackage[dvipsnames]{xcolor}
\usepackage{thmtools}
\usepackage[most]{tcolorbox}
\usepackage{titlesec}
\usepackage{float}
\usepackage{bigdelim} 
\usepackage{tikz}
\usepackage{pdfpages}
\usepackage{hyperref}
\usepackage{mathtools}
\usepackage{cancel}
\usepackage{fancyvrb}
\usepackage{listings}
%=======================
% PACKAGE SETTINGS
%=======================
\hypersetup{
    colorlinks=true,
    linkcolor=blue,
    urlcolor=blue,
    citecolor=blue
}
\usepackage{bookmark}

%=======================
% NEW COMMANDS
%=======================

% Symbols
\newcommand\R{\ensuremath{\mathbb{R}}}
\newcommand\Z{\ensuremath{\mathbb{Z}}}
\renewcommand\O{\ensuremath{\emptyset}}
\newcommand\B{\ensuremath{\mathcal{B}\ }}

% Mathematical notation
\newcommand{\inv}{^{-1}}

% Redefining symbols
\let\svlim\lim \def\lim{\svlim\limits}
\let\implies\Rightarrow
\let\impliedby\Leftarrow
\let\iff\Leftrightarrow
\let\epsilon\varepsilon

\newcommand\Tstrut{\rule{0pt}{2.6ex}}         % = `top' strut
\newcommand\Bstrut{\rule[-0.9ex]{0pt}{0pt}}   % = `bottom' strut
%=======================
% STYLING AND FORMATTING
%=======================
    % Define theorem environments

    % \newtheoremstyle{name}
    %   {space above}     % 1
    %   {space below}     % 2  
    %   {body font}       % 3
    %   {indent amount}   % 4
    %   {theorem head font} % 5
    %   {punctuation after theorem head} % 6
    %   {space after theorem head} % 7
    %   {theorem head spec} % 8
    %   {space after theorem head} % 9

% Theorem

% Definition
% \newtheoremstyle{DefStyle}
%     {10pt}
%     {10pt}%
%     {}
%     {}%
%     {\bfseries}
%     {:}%
%     { }
%     {}%
%     {}
    
% Note
\newtheoremstyle{DefStyle}
    {10pt}
    {10pt}%
    {}
    {}%
    {\bfseries}
    {:}%
    { }
    {}%
    {}
    

\theoremstyle{DefStyle}
% \theoremstyle{plain}
% \theoremstyle{definition}


\newtheorem{thm}{Theorem}
\newtheorem{exmp}{Example}
\newtheorem{defn}{Definition}
\newtheorem{note}{Note}

% \declaretheoremstyle[
%     headfont=\bfseries\sffamily, bodyfont=\normalfont,
%     % mdframed={
%     %     linewidth=2pt,
%     %     rightline=false, topline=false, bottomline=false,
%     %     linecolor=ForestGreen, backgroundcolor=ForestGreen!5,
%     % }
% ]{sdef}


\declaretheoremstyle[
    headfont=\bfseries\sffamily\color{ForestGreen!70!black}, bodyfont=\normalfont,
    mdframed={
        linewidth=2pt,
        rightline=false, topline=false, bottomline=false,
        linecolor=ForestGreen, backgroundcolor=ForestGreen!5,
    }
]{thmgreenbox}

\declaretheorem{Theorem}
% \declaretheorem[style=sdef]{sdef}
\declaretheorem[style=thmgreenbox]{greenboxtheorem}


\newtcolorbox{noot}[3][]
{
  arc=0mm,
  colback  = white,
  colframe = #2!50,
  coltitle = #2!20!black,   
  title    = {Side-note: #3},
  fonttitle=\sffamily,
  breakable,
  #1,
}

\newenvironment{sidenote}[1]{\begin{noot}{gray}{#1}}{\end{noot}}

\titleformat{\chapter}[frame]
  {\normalfont}
  {\filright
   \footnotesize
   \enspace ch. \arabic{chapter}.\enspace}
  {8pt}
  {\Large\bfseries\filcenter}
\usepackage[dotinlabels]{titletoc}
\titlecontents{chapter}[1.5em]{}{\contentslabel{2.3em}}{\hspace*{-2.3em}}{\hfill\contentspage}
\titlespacing*{\chapter} {0pt}{0pt}{40pt}     % this alters "before" spacing (the second length argument) to 0

\newcommand{\nchapter}[2]{%
    \setcounter{chapter}{#1}%
    \addtocounter{chapter}{-1}%
    \chapter{#2}
}

\newcommand{\nsection}[3]{%
    \setcounter{chapter}{#1}%
    \setcounter{section}{#2}%
    \addtocounter{section}{-1}%
    \section{#3}
}%

%%%%%%%%%%%%%%%%%%%%%%%%%%%%%%%%%%%%%%%%%%%%%%%%
\usetikzlibrary{decorations.pathreplacing,calc}
\newcommand{\ntikzmark}[2]{#2\hspace{0.5em}\tikz[overlay,remember picture,baseline=(#1.base)]{\node[inner sep=0pt] (#1) {};}}

\newcommand{\makebrace}[3]{%
    \begin{tikzpicture}[overlay, remember picture]
        \draw [decoration={brace,amplitude=0.5em},decorate]
        let \p1=(#1), \p2=(#2) in
        ({max(\x1,\x2)}, {\y1+0.8em}) -- node[right=0.6em] {#3} ({max(\x1,\x2)}, {\y2});
    \end{tikzpicture}
}

%%%%%%%%%%%%%%%%%%%%%%%%%%%%%%%%%%%%%%%%%%%%%%%%
\newenvironment{idef}[2]{%
    \par%
    \vspace{0.5em}%
    \noindent\textbf{#1:} \index{#2}
}{%
    \par%
    \vspace{0.5em}%
}

\graphicspath{{./../images/}}

\makeindex

\usepackage{subfiles}
\usepackage{titlesec}
\usepackage{booktabs}
\usepackage{longtable}
\newcommand{\sectionbreak}{\clearpage}

\makeindex[columns=1]
\setcounter{tocdepth}{0}

\title{Data Science I: Homework 1 Python Code}
\author{Neal Kuperman}
\date{\today}

\lstset{
    backgroundcolor=\color{orange!10},
    frame=single,
    basicstyle=\ttfamily\small, 
    breaklines=true,
    xleftmargin=0pt,
    numbers=left,
    numberstyle=\tiny\color{gray},
    numbersep=8pt,
    tabsize=4,
    showstringspaces=false
}

\begin{document}

\maketitle
\tableofcontents

\chapter{Helper Functions}

\section{Imports}
\begin{lstlisting}[language=Python]
    import ISLP 
    import matplotlib.pyplot as plt
    import pandas as pd
    import numpy as np
    import statsmodels.api as smf
    from IPython.display import display
    from lin_reg_plots import LinearRegDiagnostic
    import pickle
    from ucimlrepo import fetch_ucirepo 
    from sklearn.linear_model import LinearRegression, Ridge, Lasso
    from sklearn.model_selection import train_test_split
    from sklearn.preprocessing import StandardScaler
    from sklearn.metrics import mean_squared_error, r2_score
    from sklearn.model_selection import GridSearchCV
    from sklearn.feature_selection import SequentialFeatureSelector
\end{lstlisting}

\section{Summarize Function}
\begin{lstlisting}[language=Python]
    def summarize(model, vars = [], verbose = True):
        if isinstance(model.params, np.ndarray):
            params = model.params
            if len(vars) != (len(params) - 1):
                if verbose:
                    print("Warning: The number of variables does not match the number of parameters")
                vars = ["intercept"] + [f"x{i}" for i in range(len(params) - 1)]
        else:
            params = model.params.values
            vars = model.params.index.to_list()
        
        tvalues = np.round(model.tvalues, 4)
        pvalues = np.round(model.pvalues, 4)
        std_err = np.round(model.bse, 4)

        param_summary = pd.DataFrame(index=vars)
        param_summary["coef"] = params
        param_summary["t value"] = tvalues
        param_summary["p value"] = pvalues
        param_summary["std err "] = std_err
        

        r_squared = np.round(model.rsquared, 4)
        F_stat = np.round(model.fvalue, 4)  
        # model_summary = pd.DataFrame(columns=["R-squared", "F-statistic"])
        model_summary = pd.DataFrame({"R-squared": [r_squared], "F-statistic": [F_stat]}, index = ["value"])

        return param_summary, model_summary

\end{lstlisting}

\section{Leverage Functions}
\begin{lstlisting}[language=Python]
    def calc_leverage(x):
        """
        Observations with high leverage have an unusual value for xi. 
        High leverage observations tend to have a sizable impact on the 
        estimated regression line.   

        SLR:  hi = 1/n + (xi - x_bar)^2 / sum((xi - x_bar)^2)
        MLR:  Hii = diag(H), where H = X^T (X^T X)^-1 X
        """
        H = x @ np.linalg.inv(x.T @ x) @ x.T
        leverage_points = np.diag(H)
        return leverage_points

    def find_high_leverage_points(x):
        """
        Observations with high leverage have an unusual value for xi.
        hii > 2p/n
            p = number of predictors
            n = number of observations
        """
        leverage_points = calc_leverage(x)
        high_leverage_points = leverage_points > 2 * x.shape[1]/x.shape[0]
        index = np.where(high_leverage_points)[0]
        return index, leverage_points[index]
\end{lstlisting}

\section{Cook's Distance}
\begin{lstlisting}[language=Python]
    def calc_cooks_distance(x, y):
        """
        Cook's distance is a measure of the influence of an observation 
        on the estimated regression coefficients.
        
        D_i = (e_i^2 / (p * MSE)) * (h_ii / (1 - h_ii)^2)
        
        where:
            e_i = residual (y_i - y_hat_i)
            h_ii = leverage (diagonal of hat matrix)
            p = number of predictors (including intercept)
            MSE = mean squared error = SS_res / (n - p)
        """
        n, p = x.shape
        
        # Fit model
        beta = np.linalg.inv(x.T @ x) @ x.T @ y
        y_hat = x @ beta
        residuals = y - y_hat
        
        # MSE
        SS_res = np.sum(residuals**2)
        MSE = SS_res / (n - p)
        
        # Leverage
        h = calc_leverage(x)
        
        # Cook's distance
        cooks_d = (residuals**2 / (p * MSE)) * (h / (1 - h)**2)
        
        return cooks_d
\end{lstlisting}

%=============================================
\chapter{ISLP 3.10}
%=============================================

\section{Load Data}
\begin{lstlisting}[language=Python]
    carseats = ISLP.load_data('Carseats')

    quant_cols = carseats.select_dtypes(include=['number']).columns
    cat_cols = carseats.select_dtypes(include=['category']).columns

    # convert categorical columns to dummy variables. Creates a new column for each category level for each categorical column.
    carseats = pd.get_dummies(carseats, columns=cat_cols, dtype=float)
    \end{lstlisting}

    \section{(a) Multiple Regression: Sales $\sim$ Price + Urban + US}
    \begin{lstlisting}[language=Python]
    # Must add an intercept column when using statsmodels.api.OLS
    X = carseats[["Price", "Urban_Yes", "US_Yes"]]
    X.insert(0, 'Intercept', 1.0)
    y = carseats["Sales"]

    model = smf.OLS(y, X).fit()
    print(model.summary())
    \end{lstlisting}

    \section{(e) OLS with Price and US\_Yes Only}
    \begin{lstlisting}[language=Python]
    X = carseats[["Price", "US_Yes"]]
    X.insert(0, 'Intercept', 1.0)
    y = carseats["Sales"]

    model = smf.OLS(y, X).fit()
    print(model.summary())
    # print(model.summary().as_latex())
\end{lstlisting}

\section{(g) Confidence Intervals}
\begin{lstlisting}[language=Python]
    model.conf_int(alpha=0.05)
\end{lstlisting}

\section{(h) High Leverage Points and Cook's Distance}
\begin{lstlisting}[language=Python]
    index, high_leverage_points = find_high_leverage_points(X.values)
    lvg_df = pd.DataFrame({"index": index, "high_leverage_points": high_leverage_points})
    # print(lvg_df.to_latex(index=False))

    # Get influence measures
    influence = model.get_influence()
    cooks_d = influence.cooks_distance[0]  # [0] is the values, [1] is p-values


    # Plot Cook's D
    plt.stem(range(len(cooks_d)), cooks_d, markerfmt=",")
    plt.xlabel('Observation')
    plt.ylabel("Cook's Distance")
    plt.savefig("../images/3_10_h_cooks_d.png")
    plt.show()
\end{lstlisting}

%=============================================
\chapter{ISLP 3.14}
%=============================================

\section{(a) Generate Data}
\begin{lstlisting}[language=Python]
    np.random.seed(5)
    rng = np.random.default_rng(10)
    x1 = rng.uniform(0, 1, size=100)
    x2 = 0.5 * x1 + rng.normal(size=100) / 10
    y = 2 + 2 * x1 + 0.3 * x2 + rng.normal(size=100)
\end{lstlisting}

\section{(b) Correlation Between x1 and x2}
\begin{lstlisting}[language=Python]
    # Manual calculation of correlation
    x1_mean = np.mean(x1)
    x2_mean = np.mean(x2)

    numerator = np.sum((x1 - x1_mean) * (x2 - x2_mean))
    denominator = np.sqrt(np.sum((x1 - x1_mean)**2) * np.sum((x2 - x2_mean)**2))

    correlation = numerator / denominator
    print(f"Correlation between x1 and x2: {correlation:.3f}")

    plt.scatter(x1, x2)
    plt.xlabel('x1')
    plt.ylabel('x2')
    plt.title('Scatterplot of x1 and x2')
    plt.show()
\end{lstlisting}

\section{(c) OLS with x1 and x2}
\begin{lstlisting}[language=Python]
    intercept = np.ones(len(x1))
    X = np.column_stack((intercept, x1, x2))
    model = smf.OLS(y, X).fit()
    print(model.summary())
    # print(model.summary().as_latex())
\end{lstlisting}

\section{(d) OLS with x1 Only}
\begin{lstlisting}[language=Python]
    intercept = np.ones(len(x1))
    X = np.column_stack((intercept, x1))
    model = smf.OLS(y, X).fit()
    print(model.summary())
    # print(model.summary().as_latex())
\end{lstlisting}

\section{(e) OLS with x2 Only}
\begin{lstlisting}[language=Python]
    intercept = np.ones(len(x1))
    X = np.column_stack((intercept, x2))
    model = smf.OLS(y, X).fit()
    print(model.summary())
    # print(model.summary().as_latex())
\end{lstlisting}

\section{(g) New Observation Analysis}
\begin{lstlisting}[language=Python]
    x1_new = np.concatenate([x1, [0.1]])
    x2_new = np.concatenate([x2, [0.8]])
    y_new = np.concatenate([y, [6]])

    intercept = np.ones(len(x1_new))
    X_full = np.column_stack((intercept, x1_new, x2_new))
    X_x1 = np.column_stack((intercept, x1_new))
    X_x2 = np.column_stack((intercept, x2_new))

    model_full = smf.OLS(y_new, X_full).fit()
    model_x1 = smf.OLS(y_new, X_x1).fit()
    model_x2 = smf.OLS(y_new, X_x2).fit()

    print("Full Model")
    print("="*60)
    param_summary_full, model_summary_full = summarize(model_full, verbose=False)
    display(param_summary_full)
    display(model_summary_full)

    print("\nx1 Model")
    print("="*60)
    param_summary_x1, model_summary_x1 = summarize(model_x1, verbose=False)
    display(param_summary_x1)
    display(model_summary_x1)

    print("\nx2 Model")
    print("="*60)
    param_summary_x2, model_summary_x2 = summarize(model_x2, verbose=False)
    display(param_summary_x2)
    display(model_summary_x2)

    # Check for high leverage and influential points
    index, high_leverage_points = find_high_leverage_points(X_full)
    print(index)
    print(high_leverage_points)

    influential_stats = model_full.get_influence()
    cooks_D = influential_stats.cooks_distance[0] 
    influential_pts_index = np.where(cooks_D > 1)[0]
    influential_pts_vals = cooks_D[influential_pts_index]

    print(influential_pts_index)
    print(influential_pts_vals)

    cls = LinearRegDiagnostic(model_full)
    vif, fig, ax = cls()
    print(vif)
\end{lstlisting}

%=============================================
\chapter{ISLP 3.15}
%=============================================

\section{(a) Simple Linear Regression for Each Predictor}
\begin{lstlisting}[language=Python]
    bos = ISLP.load_data('Boston')

    # Get predictor columns (excluding 'crim')
    predictors = [col for col in bos.columns if col != 'crim']
    n_predictors = len(predictors)

    # Create subplot grid
    n_cols = 3
    n_rows = (n_predictors + n_cols - 1) // n_cols
    fig_1, axes_1 = plt.subplots(n_rows, n_cols, figsize=(16, 3*n_rows))
    axes_1 = axes_1.flatten()

    fig_2, axes_2 = plt.subplots(n_rows, n_cols, figsize=(16, 3*n_rows))
    axes_2 = axes_2.flatten()

    models = {}
    r_squared = []

    for i, col_name in enumerate(predictors):
        if col_name == 'crim':
            continue
        X = bos[col_name].values
        intercept = np.ones(len(X))
        X = np.column_stack((intercept, X))
        model = smf.OLS(bos['crim'], X).fit()
        models[col_name] = model
        predictions = model.predict(X)
        residual = bos['crim'] - predictions
        r_squared.append(model.rsquared)
        
        axes_1[i].scatter(bos[col_name], bos['crim'], alpha=0.5)
        axes_1[i].plot(bos[col_name].sort_values(), predictions[bos[col_name].argsort()], color='red')
        axes_1[i].set_xlabel(col_name, fontsize=14)
        axes_1[i].set_ylabel('crim', fontsize=14)
        axes_1[i].set_title(f'crim vs {col_name}', fontsize=14)
        
        axes_2[i].scatter(bos[col_name], residual, alpha=0.5)
        axes_2[i].set_xlabel(col_name, fontsize=14)
        axes_2[i].set_ylabel('residual', fontsize=14)
        axes_2[i].set_title(f'residual - {col_name}', fontsize=14)

    fig_1.tight_layout(pad=1.5, h_pad=2, w_pad=1)
    fig_1.savefig('../images/3_15_a_scatter.png', dpi=300, bbox_inches='tight')

    fig_2.tight_layout(pad=1.5, h_pad=2, w_pad=1)
    fig_2.savefig('../images/3_15_a_residuals.png', dpi=300, bbox_inches='tight')

    plt.show()

    r_squared_df = pd.DataFrame({'predictor': predictors, 'r_squared': r_squared})
    r_squared_df = r_squared_df.sort_values(by='r_squared', ascending=False)
    print(r_squared_df)
    \end{lstlisting}

    \section{(b) Multiple Regression with All Predictors}
    \begin{lstlisting}[language=Python]
    X = bos[predictors]
    X.insert(0, 'Intercept', 1.0)
    y = bos['crim']
    model = smf.OLS(y, X).fit()
    print(model.summary())

    param_summary, model_summary = summarize(model, verbose=False)
    display(param_summary)
    display(model_summary)
    \end{lstlisting}

    \section{(d) Non-linear Associations (Polynomial Regression)}
    \begin{lstlisting}[language=Python]
    n_cols = 3
    n_rows = (n_predictors + n_cols - 1) // n_cols
    fig_1, axes_1 = plt.subplots(n_rows, n_cols, figsize=(16, 3*n_rows))
    axes_1 = axes_1.flatten()

    fig_2, axes_2 = plt.subplots(n_rows, n_cols, figsize=(16, 3*n_rows))
    axes_2 = axes_2.flatten()

    models = {}
    r_squared = []

    for i, col_name in enumerate(predictors):
        if col_name == 'crim':
            continue
        X = bos[col_name].values
        X2 = X**2
        X3 = X**3
        intercept = np.ones(len(X))
        X = np.column_stack((intercept, X, X2, X3))
        _df = pd.DataFrame(X)
        _df.columns = ['intercept', col_name, f'{col_name}^2', f'{col_name}^3']
        model = smf.OLS(bos['crim'], _df).fit()
        models[col_name] = model
        predictions = model.predict(X)
        residual = bos['crim'] - predictions
        r_squared.append(model.rsquared)

        print(f"Model for {col_name}:")
        print("="*60)
        param_summary, model_summary = summarize(model, verbose=False)
        display(param_summary)
        display(model_summary)
        print("\n")

        axes_1[i].scatter(bos[col_name], bos['crim'], alpha=0.5)
        axes_1[i].plot(bos[col_name].sort_values(), predictions[bos[col_name].argsort()], color='red')
        axes_1[i].set_xlabel(col_name, fontsize=14)
        axes_1[i].set_ylabel('crim', fontsize=14)
        axes_1[i].set_title(f'crim vs {col_name}', fontsize=14)

        axes_2[i].scatter(bos[col_name], residual, alpha=0.5)
        axes_2[i].set_xlabel(col_name, fontsize=14)
        axes_2[i].set_ylabel('residual', fontsize=14)
        axes_2[i].set_title(f'residual - {col_name}', fontsize=14)

    fig_1.tight_layout(pad=1.5, h_pad=2, w_pad=1)
    fig_1.savefig('../images/3_15_d_scatter.png', dpi=300, bbox_inches='tight')

    fig_2.tight_layout(pad=1.5, h_pad=2, w_pad=1)
    fig_2.savefig('../images/3_15_d_residuals.png', dpi=300, bbox_inches='tight')

    plt.tight_layout()
    plt.show()

    r_squared_df = pd.DataFrame({'predictor': predictors, 'r_squared': r_squared})
    r_squared_df = r_squared_df.sort_values(by='r_squared', ascending=False)
    print(r_squared_df)
    # r_squared_df.to_latex(index = False)
\end{lstlisting}

%=============================================
\chapter{ESL 3.17}
%=============================================

\section{Load and Prepare Data}
\begin{lstlisting}[language=Python]
    spambase = fetch_ucirepo(id=94)
    scaler = StandardScaler()

    X = spambase.data.features
    y = spambase.data.targets

    X_train, X_test, y_train, y_test = train_test_split(
        X, y, test_size=0.3, random_state=42
    )

    X_train_scaled = scaler.fit_transform(X_train)
    X_test_scaled = scaler.transform(X_test)
    \end{lstlisting}

    \section{OLS, Ridge, and Lasso Regression}
    \begin{lstlisting}[language=Python]
    models = {
        'LS': {
            'model': LinearRegression(),
            'param_grid': {}
        },
        'ridge': {
            'model': Ridge(),
            'param_grid': {
                'alpha': np.concatenate([np.arange(0.005,10, 0.05), np.arange(10,2000,10)]), 
                "max_iter": [1000]
            }
        },
        'lasso': {
            'model': Lasso(),
            'param_grid': {
                'alpha': np.concatenate([np.arange(0.01,.1, 0.001), np.arange(1,2000,10)]), 
                'max_iter': [1000]
            }
        },
    }

    results = {
        "name": [], "model": [], "intercept": [], 
        "coefficients": [], "test_error": [], "test_R2": []
    }

    for name, model in models.items():
        print()
        print(name)
        print("="*60)
        opt_param = {}

        if model["param_grid"]:
            grid_search = GridSearchCV(
                model["model"], 
                param_grid=model["param_grid"], 
                cv=5, 
                scoring='neg_mean_squared_error'
            )
            grid_search.fit(X_train_scaled, y_train)
            opt_param = {
                "alpha": grid_search.best_params_['alpha'],
                "max_iter": grid_search.best_params_['max_iter']
            }
            
        model = model["model"].set_params(**opt_param)
        model.fit(X_train_scaled, y_train)

        test_pred = model.predict(X_test_scaled)
        test_error = mean_squared_error(y_test, test_pred)
        test_R2 = r2_score(y_test, test_pred)

        print(test_R2)
        
        results["name"].append(name)
        results["model"].append(model)
        results["intercept"].append(model.intercept_[0])
        results["coefficients"].append(model.coef_.flatten())
        results["test_error"].append(test_error)
        results["test_R2"].append(test_R2)

    # Create coefficients DataFrame
    coefs = np.column_stack([arr for arr in results["coefficients"]])
    coefs = np.vstack([results["intercept"], coefs])

    cols = X.columns.to_list()
    cols = ['intercept'] + cols
    model_names = results["name"]

    df = pd.DataFrame(coefs, index=cols, columns=model_names)
    \end{lstlisting}

    \section{Best Subset Selection}
    \begin{lstlisting}[language=Python]
    def best_subset_selection(X, y, X_train, X_test, y_train, y_test, rerun=False):
        best_subset_models = {}
        results = {}
        linear_models = {
            "model": [],
            "coefs": [],
            "intercept": [],
            "score": []
        }

        if rerun:
            for i in range(1, len(X.columns)):
                print(i)
                sfs = SequentialFeatureSelector(
                    LinearRegression(), 
                    n_features_to_select=i, 
                    direction='forward',
                )
                sfs.fit(X_train, y_train)

                # Select features from TRAIN data, fit on TRAIN
                selected_train = X_train[:, sfs.get_support()]
                selected_test = X_test[:, sfs.get_support()]
            
                selected_cols = X.columns[sfs.get_support()]
                df_train = pd.DataFrame(selected_train, columns=selected_cols)
                df_test = pd.DataFrame(selected_test, columns=selected_cols)

                lin_mod = LinearRegression()
                lin_mod.fit(df_train, y_train)

                best_subset_models[i] = sfs
                results[i] = sfs.get_support()
                linear_models["model"].append(lin_mod)
                linear_models["coefs"].append(lin_mod.coef_)
                linear_models["intercept"].append(lin_mod.intercept_)
                linear_models["score"].append(lin_mod.score(df_test, y_test))

            results = pd.DataFrame(results)
            results.index.name = 'Feature'

            save_dict = {
                "linear_models": linear_models,
                "results": results,
                "best_subset_models": best_subset_models
            }   

            with open("HW_1_ESL_3_17.pkl", "wb") as f:
                pickle.dump(save_dict, f)

        else:
            with open('HW_1_ESL_3_17.pkl', 'rb') as f:
                save_dict = pickle.load(f)
            results = save_dict["results"]
            linear_models = save_dict["linear_models"]
            best_subset_models = save_dict["best_subset_models"]

        plt.scatter(range(1, len(linear_models["score"])+1), linear_models["score"])
        plt.savefig("../images/ESL_3_17_best_subset_score.png")
        plt.show()
        return results, linear_models, best_subset_models

    # Run best subset selection
    best_subset_results, best_subset_linear_models, best_subset_models = \
        best_subset_selection(X, y, X_train, X_test, y_train, y_test, rerun=False)

    best_subset_idx = best_subset_linear_models["score"].index(
        max(best_subset_linear_models["score"])
    )
    best_subset_model = best_subset_models[best_subset_idx]

    col_mask = best_subset_model.get_support() 
    cols = X.columns[col_mask] 

    # Select features from TRAIN data, fit on TRAIN
    selected_train = X_train_scaled[:, col_mask]
    selected_test = X_test_scaled[:, col_mask]

    # Add intercept
    selected_train_with_intercept = smf.add_constant(selected_train)
    selected_test_with_intercept = smf.add_constant(selected_test)

    # Fit OLS model
    model = smf.OLS(y_train, selected_train_with_intercept).fit()

    # Create best subset df
    all_indices = df.index.to_list()
    best_subset_df = pd.DataFrame({"best_subset": model.params[1:].values}, index=cols)

    best_subset_df_all_indices = pd.DataFrame(
        {"best_subset": [0.0]*len(all_indices)}, 
        index=all_indices
    )

    shared_idx = best_subset_df.index.intersection(best_subset_df_all_indices.index)
    best_subset_df_all_indices.loc[shared_idx, "best_subset"] = \
        best_subset_df.loc[shared_idx, "best_subset"]
    best_subset_df_all_indices.loc["intercept", "best_subset"] = model.params.iloc[0]

    # Add best subset model to df of coefficients
    df["best_subset"] = best_subset_df_all_indices["best_subset"]

    # Add best subset model info to results dictionary
    test_pred = model.predict(selected_test_with_intercept)
    test_error = mean_squared_error(y_test, test_pred)
    test_R2 = r2_score(y_test, test_pred)

    results["name"].append("best subset")
    results["model"].append(model)
    results["intercept"].append(model.params.iloc[0])
    results["coefficients"].append(model.params[1:].values)
    results["test_error"].append(test_error)
    results["test_R2"].append(test_R2)
    \end{lstlisting}

    \section{Coefficient Comparison Plot}
    \begin{lstlisting}[language=Python]
    # Split features into 10 groups for readability
    n_features = len(df)
    chunk_size = (n_features + 9) // 10

    fig, axes = plt.subplots(5, 2, figsize=(14, 20))
    axes = axes.flatten()

    for i, ax in enumerate(axes):
        start = i * chunk_size
        end = min((i + 1) * chunk_size, n_features)
        
        df.iloc[start:end].plot(kind='bar', ax=ax, width=0.8, legend=False)
        
        ax.set_ylabel('Coefficient Value')
        ax.axhline(y=0, color='black', linestyle='-', linewidth=0.5)
        ax.tick_params(axis='x', rotation=45)
        
        for label in ax.get_xticklabels():
            label.set_ha('right')

    axes[0].set_title('Coefficients by Feature')

    handles, labels = axes[0].get_legend_handles_labels()
    fig.legend(handles, labels, loc='lower center', ncol=len(labels), bbox_to_anchor=(0.5, -0.02))

    plt.tight_layout()
    fig.subplots_adjust(bottom=0.08)

    plt.savefig("../images/ESL_3_17_coefficients.png", bbox_inches='tight')
    plt.show()

    # Print results
    print(df)
    print(results["test_R2"])
    print(results["test_error"])
\end{lstlisting}

\end{document}
